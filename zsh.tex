\PassOptionsToPackage{cmyk}{xcolor}
\documentclass[a4paper,landscape,columns=3]{cheatsheet}
%\usepackage[margin=3cm]{geometry}
%\usepackage[french,english]{babel}
%\usepackage{enumerate}
%\usepackage{palatino}
\usepackage{xcolor}
\definecolor{dark-gray}{gray}{0.3}
\definecolor{black-gray}{gray}{0.2}
\definecolor{violette}{RGB/cmyk}{38,23,77/.5,.7,0,.7}
\definecolor{dred}{RGB/cmyk}{230,0,0/0,1,1,.1}
\usepackage{fontspec}
\setmainfont{Chaparral Pro}[
        Path            = ../../source/fonts/ ,
        Extension       = .otf ,
        UprightFont     = *-Regular ,
        BoldFont        = *-Bold ,
        ItalicFont      = *-Italic ,
        BoldItalicFont  = *-BoldIt ,
        % Light           = *-Light ,
        % LightItalic     = *-LightItalic ,
        Ligatures       = Common ,
        Numbers         = OldStyle ,
        Numbers         = Proportional
]
\setsansfont{Amplitude}[
        Path            = ../../source/fonts/ ,
        Extension       = .otf ,
        UprightFont     = *-Regular ,
        BoldFont        = *-Bold ,
%        BoldItalicFont  = *-BoldIt ,
        % Light           = *-Light ,
        % LightItalic     = *-LightItalic ,
%        Ligatures       = Common ,
%        Numbers         = OldStyle ,
        Numbers         = Proportional
]
\setmonofont{Droid Sans Mono}[%
	Color		= dred ,
	Scale		= 0.95 
]
\newfontfamily{\dsm}{Droid Sans Mono}[%
	Scale		= 0.95
]
\newfontfamily{\chpl}{Chaparral Pro}[
        Path            = ../../source/fonts/ ,
        Extension       = .otf ,
        UprightFont     = *-Bold ,
%        ItalicFont      = *-LightIt ,
%	Bold	Font		= *-Bold ,
        Ligatures       = Common ,
        Numbers         = OldStyle ,
        Numbers         = Monospaced  ,
        Color		= dred 
]
%\usepackage[bf,compact]{titlesec}
\usepackage[rm]{titlesec}
\titleformat{\section}{\ttfamily\Large}{\thesection}{1.77em}{}
%\titleformat{\section}{\bfseries\color{dred}}{\thesection}{1.77em}{}
%\titlespacing*{\paragraph}{0pt}{3.25ex plus 1ex minus .2ex}{1em}
%\titlespacing{\section}{0pt}{\baselineskip}{.5\baselineskip}
\usepackage{fancyhdr}
\pagestyle{fancy}
\fancyhf{}
\renewcommand{\footrulewidth}{0pt}
\renewcommand{\headrulewidth}{0pt}
\fancyfoot[R]{\chpl{\textcolor{dred}{\thepage}}}
\usepackage{enumitem}
\usepackage{kvsetkeys}
\usepackage{keyval}
\usepackage{hyperref}
\hypersetup{%
pdfpagemode=UseNone,
pdftitle={Oh My Zsh: cheat sheet},
pdfauthor={Patrick O'Donovan},
pdfcreator={Pandoc},
pdfstartview=FitH,
pdfdisplaydoctitle=true,
colorlinks=true,
urlcolor={violette}%
}

\begin{document}

\sffamily

\section{Oh My Zsh}

\bigskip

\section{Commands}

\texttt{tab}\\
Open the current directory in a new tab

\texttt{tabs}\\
Create a new tab in the current directory

\texttt{take}\\
Create a new directory and switch to it; will create intermediate directories as required

\texttt{x} or \texttt{extract}\\
Extract an archive

\texttt{extract -r}\\
Extract and remove original archive file

\texttt{ESC ESC}\\
Puts {\dsm{sudo}} in front of the current command, or the last one if the command line is empty.

\texttt{zsh\_stats}\\
Get list of top twenty commands

\texttt{upgrade\_oh\_my\_zsh}

\texttt{exec zsh}\\
Apply changes made to \texttt{.zshrc}

\texttt{ofd}\\
 Open the current directory in a Finder window

\texttt{pfd}\\
 Return the path of the frontmost Finder window

\texttt{pfs}\\
 Return the current Finder selection

\texttt{cdf }\\
{\dsm{cd}} to the current Finder directory

\texttt{pushdf}\\
{\dsm{pushd}} to the current Finder directory

\texttt{man-preview}\\
 Open a specified man page in Preview app

\texttt{showfiles}\\
 Show hidden files

\texttt{hidefiles}\\
 Hide the hidden files
 
\columnbreak

\section{Aliases}

\texttt{alias}\\
List all aliases

\texttt{alias -L}\\
Print each alias in the form of calls to alias

\texttt{alias -g}\\
list or define global aliases

\texttt{\_}\\
{\dsm{sudo}}

\texttt{brews}\\
{\dsm{brew list -1}}

\texttt{bubc}\\
{\dsm{brew upgrade \&\& brew cleanup}}

\texttt{bubo}\\
{\dsm{brew update \&\& brew outdated}}

\texttt{bubu}\\
{\dsm{bubo \&\& bubc}} 

\texttt{h}\\
{\dsm{history}}

\texttt{hsi}\\
{\dsm{hs -i}}

\texttt{l}\\
{\dsm{ls -lah}}

\texttt{la}\\
{\dsm{ls -lAh}}

\texttt{ll}\\
{\dsm{ls -lh}}

\texttt{ls}\\
{\dsm{ls -G}}

\texttt{lsa}\\
{\dsm{ls -lah}}

\texttt{please}\\
{\dsm{sudo}}

\texttt{po}\\
{\dsm{popd}}

\texttt{pu}\\
{\dsm{pushd}}

\texttt{ql}\\
{\dsm{qlmanage -p}}

\texttt{quicklook}\\
{\dsm{qlmanage -p}}

\section{Directories}

\texttt{wd}\\
Warp directory

\texttt{wd add test}\\
Add warp point to current working directory

\texttt{wd ..}\\
Warp back to the previous directory

\texttt{wd show}\\
List warp points to current directory

\texttt{wd list}\\
List all warp points

\texttt{..}\\
{\dsm{cd ..}}

\texttt{...}\\
{\dsm{cd ../..}}

\texttt{....}\\
{\dsm{cd ../../..}}

\texttt{.....}\\
{\dsm{cd ../../../..}}

\texttt{/}\\
{\dsm{cd}} /

\texttt{\textasciitilde}\\
{\dsm{cd} \textasciitilde

\texttt{md}\\
{{\dsm{mkdir -p}}

\texttt{rd}\\
{{\dsm{rmdir}}

\texttt{d}\\
{\dsm{dirs -v}} \sffamily (lists last used directories)

\texttt{lwd}\\
Jump to last working directory (automatically called for new shells)

\section{Searches}

\texttt{sp}\\
Startpage

\texttt{ddg}\\
Duckduckgo

\texttt{ecosia}

\texttt{github}

Via Duckduckgo

\ttfamily

image \qquad map \qquad news

wiki \qquad youtube

\columnbreak

\sffamily

\section{Plugins}

\ttfamily

copyfile

dirhistory 

extract

forklift

history

last-working-dir

osx

per-directory-history {\dsm{(ctrl-g)}}

shrink-path

sudo

wd

web-search

\end{document}